\documentclass{bmvc2k}

%\bmvcreviewcopy{??}

\title{An Investigation of Colour Constancy for Image Classification}

\addauthor{Uzairu Abubakar}{uzairu.abubakar@etu.univ-st-etienne.fr}{1}
\addauthor{Niklas Bieck}{niklas.bieck@etu.univ-st-etienne.fr}{1}
\addauthor{Fabiano Maia Junior Manschein}{fabiano.junior.maia.manschein@etu.univ-st-etienne.fr}{1}
\addauthor{Yuya Takagi}{yuya.takagi@etu.univ-st-etienne.fr}{1}
\addauthor{Damien Muselet}{damien.muselet@univ-st-etienne.fr}{1}

\addinstitution{
    Université Jean Monnet Saint-Étienne \\
    10, rue Tréfilerie – CS 82301 \\
    42023 Saint-Etienne Cedex 2 \\
    France
}

\runninghead{Abubakar, Bieck, Manschein, Takagi, Muselet}{Colour Constancy Classification}

\begin{document}

\maketitle

\begin{abstract}
Colour Constancy, attempting to determine the colour of the light in a particular image, and then 
modifying the colour to match what it would have been under some reference illuminant, is a widely studied topic.
One of the main claims that is made as to whether this is a useful endeavor is that it could aid in the training
of other image processing, as the light colour is removed as a point of variation that needs to be learned.

This seems to just be taken as accepted fact and has never actually been empirically verified. In this paper, we
seek to rectify this situation by comparing the performance and training times of image classification models on 
a corpus of images of flowers (chosen because of the importance of colour in their identification), with or without
having applied Colour Constancy methods as preprocessing.
\end{abstract}

\section{Introduction}

\section{Related Work}

As we do not propose any kind of new algorithm, we will make use of existing solutions for both
the computation of the colour corrected image, and the actual classification of the images.
On the side of colour constancy, we have found the follwing approaches, which include code/pretrained models:
\cite{hu2017fc} {\it find more here}.

There are two datasets of labeled images of flowers we can use to test: \cite{Nilsback06} and \cite{Nilsback08}

\bibliography{references}
\end{document}