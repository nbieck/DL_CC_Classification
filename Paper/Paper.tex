\documentclass[runningheads]{llncs}
\bibliographystyle{splncs04}

\usepackage{graphicx}
\usepackage{tikz}
\usepackage{pgfplots}
\usepackage{pgfplotstable}
\pgfplotstableset{col sep=comma}
\usetikzlibrary{plotmarks}

\definecolor{ourorange}{RGB}{230,159,0}
\definecolor{ourblue}{RGB}{86,180,233}
\definecolor{ourgreen}{RGB}{0,158,115}
\definecolor{ouryellow}{RGB}{240,228,66}
\definecolor{ourdarkblue}{RGB}{0,114,178}
\definecolor{ourdarkorange}{RGB}{213,94,0}
\definecolor{ourpurple}{RGB}{204,121,167}

\pgfplotsset{
    cycle list={
        {ourorange, mark=*},
        {ourblue, mark=+},
        {ourgreen, mark=x},
        {ourpurple, mark=o},
        {ourdarkblue, mark=square},
        {ourdarkorange, mark=triangle},
        {outyellow, mark=diamond},
    },
    every mark/.append style={solid},
}

\graphicspath{ {./images/} }

\begin{document}

\title{An Investigation of Color Constancy for Image Classification}
\titlerunning{Color Constancy Classification}

\author{
    Uzairu Abubakar\inst{1} \and 
    Niklas Bieck\inst{1} \and 
    Fabiano Junior Maia Manschein\inst{1} \and 
    Yuya Takagi\inst{1} \and 
    Damien Muselet\inst{1}
}
\authorrunning{U. Abubakar et al.}

\institute{
    Université Jean Monnet Saint-Étienne, 10, rue Tréfilerie – CS 82301, 42023 Saint-Etienne Cedex 2, France
    \email{https://www.univ-st-etienne.fr/fr/index.html}
}



\maketitle

\begin{abstract}
Color Constancy, attempting to determine the color of the light in a particular image, and then 
modifying the color to match what it would have been under some reference illuminant, is a widely studied topic.
One of the main claims that is made as to whether this is a useful endeavor is that it could aid in the training
of other image processing, as the light color is removed as a point of variation that needs to be learned.

We seek to validate this point by comparing the performance and training times of image classification models on 
a corpus of images of flowers (chosen because of the importance of colour in their identification), with or without
having applied Color Constancy methods as preprocessing.
\end{abstract}

% Here's my suggested structure for the paper. Subject to change, specially titles.
\section{Introduction}

\gls{CC} is a natural part of the human visual system that lets us recognize that a color
stays the same even under changing illumination. Letting computers do this same thing and then
modify the image color to represent their appearance under a neutral light source is a well-established
and on-going area of research. One of the presumed benefits of doing this is improved performance
of recognition tasks that rely to a large part on color, as it removes the need for the classification
model to learn to recognize objects under changing lighting conditions.

While some effort has been made to investigate how \gls{CC} correction can aid in video tracking
\cite{Agarwal2006}, as far as we can tell, no investigation specific to the field of object
classification has, as of yet, been performed. In this paper, we will do just that, using the
classification of flowers as a sample task, since this is clearly an area where color is of large
importance to a successful classification.

% Todo:
%! 1- lots of use of 'this'. Can the sentence be rewritten to be clearer?
%! 2- the use of terms such as 'as far as we know' and 'clearly' is dangerous. Alternatives?
%! 3- 'While some effort has been made...' -> 'While efforts to investigate... have been made, as far as...'  % Introduction
\section{Related Work}

Investigations into the per-se accuracy of color constancy methods are quite common, as an analysis will usually be a part of each new proposed method.
In addition, more direct surveys without a proposed new method have also been performed \cite{barnard2002comparison1} \cite{barnard2002comparison2}.

Agarwal et al. \cite{Agarwal2006} performed an investigation into not just the performance of different color constancy methods, but also their usefulness
in the context of video tracking tasks.  % Previous related studies on applying CC as preprocessing
% This section is for the experiment part of the paper, where we talk about the
% pipeline, datasets, and the models we used.

% Suggested structure:
% 1- Experiment setup/structure (pipeline)
% 2- Datasets
% 3- CC models
% 4- CLF models

\section{Experiment}

The goal of the experiment is to evaluate the effects of \gls{CC} as a pre-processing step for the cases
of simple and complex datasets and classification models.
It consists of training and evaluation on all variations over ten trials.

For this purpose, we use two flower datasets of different complexities with 17 and 102 classes each. Our basis for comparison
are the results obtained from training and evaluating on the datasets without \gls{CC} nor batch normalization.
We then process the datasets through the different \gls{CC} methods and the batch normalization to obtain results
for the variations. For the classification of flower images, we use a fine-tuned VGG16 model and our own \gls{CNN} trained from scratch.
Figure \ref{fig:experiment_pipeline} presents an overview of the experiment.

\begin{figure}[ht]
    \centering
    \includegraphics[width=\textwidth]{images/experiment_pipeline.png}
    \caption{Experiment overview.}
    \label{fig:experiment_pipeline}
\end{figure}

\subsection{Datasets}

We used the publicly available Oxford 17 \cite{Nilsback06} and Oxford 102 \cite{Nilsback08} flower datasets.
Oxford 17 contains 1360 images, and 80 images per class. Oxford 102 contains 8189 images, with between 40 and 258 images per class.

\subsection{Color Constancy}

\begin{figure}[ht]
    \centering
    \begin{tabular}{c|cccc}
        \includegraphics[width=0.165\textwidth]{cc_demo/flower001_base.png}       &
        \includegraphics[width=0.165\textwidth]{cc_demo/flower001_whitePatch.png} &
        \includegraphics[width=0.165\textwidth]{cc_demo/flower001_greyWorld.png}  &
        \includegraphics[width=0.165\textwidth]{cc_demo/flower001_grayEdge.png}   &
        \includegraphics[width=0.165\textwidth]{cc_demo/flower001_fc4.png}                                \\
        (a)                                                                       & (b) & (c) & (d) & (e) \\
        \includegraphics[width=0.165\textwidth]{cc_demo/flower268_base.png}       &
        \includegraphics[width=0.165\textwidth]{cc_demo/flower268_whitePatch.png} &
        \includegraphics[width=0.165\textwidth]{cc_demo/flower268_greyWorld.png}  &
        \includegraphics[width=0.165\textwidth]{cc_demo/flower268_grayEdge.png}   &
        \includegraphics[width=0.165\textwidth]{cc_demo/flower268_fc4.png}                                \\
        (f)                                                                       & (g) & (h) & (i) & (j)
    \end{tabular}
    \caption{A comparison of color constancy algorithms: (a) and (f): Original image.
        (b) and (g): White Patch. (c) and (h): Grey-World.
        (d) and (i): Grey-Edge. (e) and (j): FC\textsuperscript{4}}
    \label{fig:cc_comparison}
\end{figure}

In order to have a wider spread of examined methods, we both make use of state-of-the art
learning based methods, as well as classical simpler methods
like White-Patch, Grey-World \cite{EbnerConstancy} and Grey-Edge \cite{van2005color}.
A comparison of these can be found in Figure \ref{fig:cc_comparison}.

\subsubsection{Statistical Methods}

In the grey world algorithm, the assumption is made that the average reflectance of the scene should
be a shade of grey. We can therefore infer that any deviation of the average color from this grey tone stems from
the illumination in the scene. By simply dividing this out, we receive a color-corrected image.

White-Patch is a close relative of this method, where instead we assume that the brightest spot in the image (for each channel)
is representative of the overall light color.

In the grey-edge hypothesis we instead assume that the average image gradient can be used as an indication of the light color.

All of these algorithms were reimplemented by us. Notably, based on the recommendation made by Ebner \cite{EbnerConstancy},
after color adjustment we rescale all results such that the top 5\% of values (across all channels) will be clipped to
the maximum intensity of 1.

\subsubsection{Learning Based Model}

We use the FC\textsuperscript{4} model\cite{hu2017fc} as our representative for the deep learning based models. This is a considerably more complicated approach
than the statistical methods listed above, as it makes use of deep convolutional networks.

We chose FC\textsuperscript{4}, as it can provide us with high quality outputs, while still being reasonably simple to use within our framework. While newer
models might perform slightly better when evaluated purely on their results, we judged that FC\textsuperscript{4} should already provide a sufficiently large\
delta to the statistical methods to give an indication of whether this approach is worth pursuing.

\subsection{Classification models}

In this study, two distinct classification models were employed: a Pre-trained VGG16 and a custom-built network developed by us.
To ensure a standardized approach for fair comparison and evaluation, certain key methodologies were incorporated.
Specifically, we applied consistent RMSprop optimizer and Sparse cross entropy loss function across both models.
Given that we have 17 and 102 flower categories, this loss function is suitable for our multi-class classification problem. Moreover, to explore different optimization paths, we set the learning rate
for our custom model as 1e-3 and 1e-4 for the Pre-trained VGG16 model.

\subsubsection{VGG16}

The VGG16 model, well-known for its simplicity and effectiveness in image classification task, is used as a feature extractor in our study. Pre-trained on the ImageNet dataset, it effectively captures high-level features.
By employing transfer learning, we take advantage of the pre-trained weights of the VGG16 model that were trained on the large-scale ImageNet dataset.
The last four layers are fine-tuned for our flower classification, while the first layers remain frozen to preserve latent features. The final layer is replaced with a new fully connected layer of 17 neurons and SoftMax activation function.
And a flatten layer that converts the output into a one-dimensional feature vector, followed by a dense layer of 1024 neurons and a dropout layer to prevent overfitting.

\subsubsection{Our Implemented Classifier}

As for our custom classifier,  The architecture of our implemented network includes a sequence of layers. Which essentially follows a common Convolutional Neural Network Architecture (CNN)
starting with convolutional and pooling layers for feature extraction, followed by fully connected layers for classification.
The ReLU activation function introduces non-linearity, and the dropout layer helps regularize the network.



    {\color{red} This still needs rework.}  % The experiment: pipeline, datasets, CC and CLF models
\section{Results}

%LaTeX magic that lets me create a new column in a table with the format a \pm b
\def\pgfmathprintpmnumber#1#2{%
    \pgfmathfloatparsenumber{\thisrow{#1}}%
    \let\valueAvg=\pgfmathresult
    \pgfmathfloatparsenumber{\thisrow{#2}}%
    \let\valueStd=\pgfmathresult
    \edef\valueAvg{\noexpand\pgfmathprintnumber[std, precision=2]{\valueAvg}}%
    \edef\valueStd{\noexpand\pgfmathprintnumber[std, precision=2]{\valueStd}}%
    \toks0=\expandafter{\valueAvg}%
    \toks1=\expandafter{\valueStd}%
    \edef\value{\the\toks0$\pm$\the\toks1}%
}

%load the data for our model with 17 flowers
\pgfplotstableread{data/ours_17_flowers_summary.csv}{\ourssmallsummary}
{\scriptsize
\pgfplotstabletypeset[
    col sep=comma,
    columns={Algorithm, train_time, test_time, 
        train_loss, val_loss, test_acc},
    column type = c,
    columns/Algorithm/.style={string type, column name=},
    columns/train_time/.style = {string type, column name ={Training (s)}},
    create on use/train_time/.style={%
        create col/assign/.code={%
            \pgfmathprintpmnumber{train_time_avg}{train_time_std}
            \pgfkeyslet{/pgfplots/table/create col/next content}\value
        }
    },
    columns/test_time/.style = {string type, column name ={Testing (s)}},
    create on use/test_time/.style={%
        create col/assign/.code={%
            \pgfmathprintpmnumber{test_time_avg}{test_time_std}
            \pgfkeyslet{/pgfplots/table/create col/next content}\value
        }
    },
    columns/train_loss/.style = {string type, column name ={Train Loss}},
    create on use/train_loss/.style={%
        create col/assign/.code={%
            \pgfmathprintpmnumber{train_loss_avg}{train_loss_std}
            \pgfkeyslet{/pgfplots/table/create col/next content}\value
        }
    },
    columns/val_loss/.style = {string type, column name ={Val Loss}},
    create on use/val_loss/.style={%
        create col/assign/.code={%
            \pgfmathprintpmnumber{val_loss_avg}{val_loss_std}
            \pgfkeyslet{/pgfplots/table/create col/next content}\value
        }
    },
    columns/test_acc/.style = {string type, column name ={Test Acc}},
    create on use/test_acc/.style={%
        create col/assign/.code={%
            \pgfmathprintpmnumber{test_acc_avg}{test_acc_std}
            \pgfkeyslet{/pgfplots/table/create col/next content}\value
        }
    },
    every head row/.style = {before row=\hline, after row=\hline},
    every last row/.style = {after row=\hline},
    every column/.style = {column type/.add={|}{}},
    every last column/.style = {column type/.add={}{|}},
]{\ourssmallsummary}
}

\begin{figure}
    \centering
    \begin{tabular}{cc}
        \begin{tikzpicture}
            \begin{axis}[
                title = Training Time,
                ylabel = Time/s,
                ybar,
                xtick=data,
                xticklabels from table = {\ourssmallsummary}{Algorithm},
                x tick label style = {rotate=45, anchor=east},
                error bars/y dir=both,
                error bars/y explicit,
                error bars/error bar style={black},
                ymin=0,
                width=0.49\textwidth,
                bar width=6pt,
            ]
                \addplot+[
                    draw=black,
                    fill=ourblue,
                ] table[
                    col sep=comma,
                    x expr=\coordindex,
                    y=train_time_avg,
                    y error=train_time_std,
                ] {\ourssmallsummary};
                \addplot+[
                    draw=black,
                    fill=ourorange,
                ] table[
                    col sep=comma,
                    x expr=\coordindex,
                    y=train_time,
                ] {data/vgg16_17_flowers_summary.csv};
            \end{axis}
        \end{tikzpicture}&
        \begin{tikzpicture}
            \begin{axis}[
                title = Testing Time,
                ybar,
                xtick=data,
                xticklabels from table = {\ourssmallsummary}{Algorithm},
                x tick label style = {rotate=45, anchor=east},
                error bars/y dir=both,
                error bars/y explicit,
                error bars/error bar style={black},
                ylabel=Time/s,
                no markers,
                ymin=0,
                width=0.49\textwidth,
                bar width=6.pt,
            ]
                \addplot+[
                    draw=black,
                    fill=ourblue,
                ] table[
                    col sep=comma,
                    x expr=\coordindex,
                    y=test_time_avg,
                    y error=test_time_std,
                ] {\ourssmallsummary};
                \addplot+[
                    draw=black,
                    fill=ourorange,
                ] table[
                    col sep=comma,
                    x expr=\coordindex,
                    y=test_time,
                ] {data/vgg16_17_flowers_summary.csv};
            \end{axis}
        \end{tikzpicture}\\
        \begin{tikzpicture}
            \begin{axis}[
                title = Training Loss,
                ylabel = Loss,
                ybar,
                xtick=data,
                xticklabels from table = {\ourssmallsummary}{Algorithm},
                x tick label style = {rotate=45, anchor=east},
                error bars/y dir=both,
                error bars/y explicit,
                error bars/error bar style={black},
                ymin=0,
                width=0.49\textwidth,
                bar width=6pt,
            ]
                \addplot+[
                    draw=black,
                    fill=ourblue,
                ] table[
                    col sep=comma,
                    x expr=\coordindex,
                    y=train_loss_avg,
                    y error=train_loss_std,
                ] {\ourssmallsummary};
                \addplot+[
                    draw=black,
                    fill=ourorange,
                ] table[
                    col sep=comma,
                    x expr=\coordindex,
                    y=train_loss,
                ] {data/vgg16_17_flowers_summary.csv};
            \end{axis}
        \end{tikzpicture}&
        \begin{tikzpicture}
            \begin{axis}[
                title = Validation Loss,
                ylabel = Loss,
                ybar,
                xtick=data,
                xticklabels from table = {\ourssmallsummary}{Algorithm},
                x tick label style = {rotate=45, anchor=east},
                error bars/y dir=both,
                error bars/y explicit,
                error bars/error bar style={black},
                ymin=0,
                width=0.49\textwidth,
                bar width=6pt,
            ]
                \addplot+[
                    draw=black,
                    fill=ourblue,
                ] table[
                    col sep=comma,
                    x expr=\coordindex,
                    y=val_loss_avg,
                    y error=val_loss_std,
                ] {\ourssmallsummary};
                \addplot+[
                    draw=black,
                    fill=ourorange,
                ] table[
                    col sep=comma,
                    x expr=\coordindex,
                    y=val_loss,
                ] {data/vgg16_17_flowers_summary.csv};
            \end{axis}
        \end{tikzpicture}\\
        \begin{tikzpicture}
            \begin{axis}[
                title = Test Accuracy,
                ylabel = Loss,
                ybar,
                xtick=data,
                xticklabels from table = {\ourssmallsummary}{Algorithm},
                x tick label style = {rotate=45, anchor=east},
                error bars/y dir=both,
                error bars/y explicit,
                error bars/error bar style={black},
                ymin=0,
                width=0.49\textwidth,
                bar width=6pt,
            ]
                \addplot+[
                    draw=black,
                    fill=ourblue,
                ] table[
                    col sep=comma,
                    x expr=\coordindex,
                    y=test_acc_avg,
                    y error=test_acc_std,
                ] {\ourssmallsummary};
                \addplot+[
                    draw=black,
                    fill=ourorange,
                ] table[
                    col sep=comma,
                    x expr=\coordindex,
                    y=test_acc,
                ] {data/vgg16_17_flowers_summary.csv};
            \end{axis}
        \end{tikzpicture}
    \end{tabular}
    \caption{Comparison a comparison of the different color constancy methods on the 17 flower dataset using our model (blue) and VGG16 (orange).}
    \label{fig:comparison_17_flowers}
\end{figure}

%load all tables for the following graphs so we don't need to load the file for each one
\pgfplotstableread{data/train_loss_17_flowers.csv}\ourssmalltrainloss
\pgfplotstableread{data/train_accuracy_17_flowers.csv}\ourssmalltrainacc
\pgfplotstableread{data/validation_loss_17_flowers.csv}\ourssmallvalloss
\pgfplotstableread{data/validation_accuracy_17_flowers.csv}\ourssmallvalacc

\def\historygraphheight{4.395cm}

\begin{figure}
    %We use a centered table with a single right-aligned column to ensure that the graphs
    %visually lign up while still being centered on the page
    \centering
    \begin{tabular}{r}
        \begin{tikzpicture}
            \begin{axis}[
                title = Training Loss over Time,
                mark repeat=1,
                xlabel = Epoch,
                ylabel = Loss,
                width=0.95\textwidth,
                height=\historygraphheight,
                each nth point={4},
                legend columns=2,
            ]
                \addplot+[
                    smooth,
                ] table[
                    x = Epoch,
                    y = Base,
                ] {\ourssmalltrainloss};
                \addlegendentry{Base}
                \addplot+[
                    smooth,
                ] table[
                    x = Epoch,
                    y = BatchNorm,
                ] {\ourssmalltrainloss};
                \addlegendentry{Batch Norm}
                \addplot+[
                    smooth,
                ] table[
                    x = Epoch,
                    y = FC4,
                ] {\ourssmalltrainloss};
                \addlegendentry{FC\textsuperscript{4}}
                \addplot+[
                    smooth,
                ] table[
                    x = Epoch,
                    y = WhitePatch,
                ] {\ourssmalltrainloss};
                \addlegendentry{White Patch}
                \addplot+[
                    smooth,
                ] table[
                    x = Epoch,
                    y = GreyEdge,
                ] {\ourssmalltrainloss};
                \addlegendentry{Grey Edge}
                \addplot+[
                    smooth,
                ] table[
                    x = Epoch,
                    y = GreyWorld,
                ] {\ourssmalltrainloss};
                \addlegendentry{Grey World}
            \end{axis}
        \end{tikzpicture}\\
        \begin{tikzpicture}
            \begin{axis}[
                title = Validation Loss over Time,
                mark repeat=1,
                xlabel = Epoch,
                ylabel = Loss,
                width=0.95\textwidth,
                height=\historygraphheight,
                each nth point={4},
                legend columns=2,
            ]
                \addplot+[
                    smooth,
                ] table[
                    x = Epoch,
                    y = Base,
                ] {\ourssmallvalloss};
                \addlegendentry{Base}
                \addplot+[
                    smooth,
                ] table[
                    x = Epoch,
                    y = BatchNorm,
                ] {\ourssmallvalloss};
                \addlegendentry{Batch Norm}
                \addplot+[
                    smooth,
                ] table[
                    x = Epoch,
                    y = FC4,
                ] {\ourssmallvalloss};
                \addlegendentry{FC\textsuperscript{4}}
                \addplot+[
                    smooth,
                ] table[
                    x = Epoch,
                    y = WhitePatch,
                ] {\ourssmallvalloss};
                \addlegendentry{White Patch}
                \addplot+[
                    smooth,
                ] table[
                    x = Epoch,
                    y = GreyEdge,
                ] {\ourssmallvalloss};
                \addlegendentry{Grey Edge}
                \addplot+[
                    smooth,
                ] table[
                    x = Epoch,
                    y = GreyWorld,
                ] {\ourssmallvalloss};
                \addlegendentry{Grey World}
            \end{axis}
        \end{tikzpicture}\\
        \begin{tikzpicture}
            \begin{axis}[
                title = Training Accuracy over Time,
                mark repeat=1,
                xlabel = Epoch,
                ylabel = Accuracy,
                legend pos=south east,
                width=0.95\textwidth,
                height=\historygraphheight,
                each nth point={4},
                legend columns=2,
            ]
                \addplot+[
                    smooth,
                ] table[
                    x = Epoch,
                    y = Base,
                ] {\ourssmalltrainacc};
                \addlegendentry{Base}
                \addplot+[
                    smooth,
                ] table[
                    x = Epoch,
                    y = BatchNorm,
                ] {\ourssmalltrainacc};
                \addlegendentry{Batch Norm}
                \addplot+[
                    smooth,
                ] table[
                    x = Epoch,
                    y = FC4,
                ] {\ourssmalltrainacc};
                \addlegendentry{FC\textsuperscript{4}}
                \addplot+[
                    smooth,
                ] table[
                    x = Epoch,
                    y = WhitePatch,
                ] {\ourssmalltrainacc};
                \addlegendentry{White Patch}
                \addplot+[
                    smooth,
                ] table[
                    x = Epoch,
                    y = GreyEdge,
                ] {\ourssmalltrainacc};
                \addlegendentry{Grey Edge}
                \addplot+[
                    smooth,
                ] table[
                    x = Epoch,
                    y = GreyWorld,
                ] {\ourssmalltrainacc};
                \addlegendentry{Grey World}
            \end{axis}
        \end{tikzpicture}\\
        \begin{tikzpicture}
            \begin{axis}[
                title = Validation Accuracy over Time,
                mark repeat=1,
                xlabel = Epoch,
                ylabel = Accuracy,
                legend pos=south east,
                width=0.95\textwidth,
                height=\historygraphheight,
                each nth point={4},
                legend columns=2,
            ]
                \addplot+[
                    smooth,
                ] table[
                    x = Epoch,
                    y = Base,
                ] {\ourssmallvalacc};
                \addlegendentry{Base}
                \addplot+[
                    smooth,
                ] table[
                    x = Epoch,
                    y = BatchNorm,
                ] {\ourssmallvalacc};
                \addlegendentry{Batch Norm}
                \addplot+[
                    smooth,
                ] table[
                    x = Epoch,
                    y = FC4,
                ] {\ourssmallvalacc};
                \addlegendentry{FC\textsuperscript{4}}
                \addplot+[
                    smooth,
                ] table[
                    x = Epoch,
                    y = WhitePatch,
                ] {\ourssmallvalacc};
                \addlegendentry{White Patch}
                \addplot+[
                    smooth,
                ] table[
                    x = Epoch,
                    y = GreyEdge,
                ] {\ourssmallvalacc};
                \addlegendentry{Grey Edge}
                \addplot+[
                    smooth,
                ] table[
                    x = Epoch,
                    y = GreyWorld,
                ] {\ourssmallvalacc};
                \addlegendentry{Grey World}
            \end{axis}
        \end{tikzpicture}
    \end{tabular}
    \caption{For each algorithm, the development of training and validation loss and accuracy are shown 
    for the trial ending on the lowest validation loss respectively.}
    \label{fig:ours_17_flowers_history}
\end{figure}

%load the data for our model with 102 flowers
\pgfplotstableread{data/ours_102_flowers_summary.csv}{\ourslargesummary}
{\scriptsize
\pgfplotstabletypeset[
    col sep=comma,
    columns={Algorithm, train_time, test_time, 
        train_loss, val_loss, test_acc},
    column type = c,
    columns/Algorithm/.style={string type, column name=},
    columns/train_time/.style = {string type, column name ={Training (s)}},
    create on use/train_time/.style={%
        create col/assign/.code={%
            \pgfmathprintpmnumber{train_time_avg}{train_time_std}
            \pgfkeyslet{/pgfplots/table/create col/next content}\value
        }
    },
    columns/test_time/.style = {string type, column name ={Testing (s)}},
    create on use/test_time/.style={%
        create col/assign/.code={%
            \pgfmathprintpmnumber{test_time_avg}{test_time_std}
            \pgfkeyslet{/pgfplots/table/create col/next content}\value
        }
    },
    columns/train_loss/.style = {string type, column name ={Train Loss}},
    create on use/train_loss/.style={%
        create col/assign/.code={%
            \pgfmathprintpmnumber{train_loss_avg}{train_loss_std}
            \pgfkeyslet{/pgfplots/table/create col/next content}\value
        }
    },
    columns/val_loss/.style = {string type, column name ={Val Loss}},
    create on use/val_loss/.style={%
        create col/assign/.code={%
            \pgfmathprintpmnumber{val_loss_avg}{val_loss_std}
            \pgfkeyslet{/pgfplots/table/create col/next content}\value
        }
    },
    columns/test_acc/.style = {string type, column name ={Test Acc}},
    create on use/test_acc/.style={%
        create col/assign/.code={%
            \pgfmathprintpmnumber{test_acc_avg}{test_acc_std}
            \pgfkeyslet{/pgfplots/table/create col/next content}\value
        }
    },
    every head row/.style = {before row=\hline, after row=\hline},
    every last row/.style = {after row=\hline},
    every column/.style = {column type/.add={|}{}},
    every last column/.style = {column type/.add={}{|}},
]{\ourslargesummary}
}

\begin{figure}
    \centering
    \begin{tabular}{cc}
        \begin{tikzpicture}
            \begin{axis}[
                title = Training Time,
                ylabel = Time/s,
                ybar,
                xtick=data,
                xticklabels from table = {\ourslargesummary}{Algorithm},
                x tick label style = {rotate=45, anchor=east},
                error bars/y dir=both,
                error bars/y explicit,
                error bars/error bar style={black},
                ymin=0,
                width=0.49\textwidth,
                bar width=6pt,
            ]
                \addplot+[
                    draw=black,
                    fill=ourblue,
                ] table[
                    col sep=comma,
                    x expr=\coordindex,
                    y=train_time_avg,
                    y error=train_time_std,
                ] {\ourslargesummary};
                \addplot+[
                    draw=black,
                    fill=ourorange,
                ] table[
                    col sep=comma,
                    x expr=\coordindex,
                    y=train_time,
                ] {data/vgg16_17_flowers_summary.csv};
            \end{axis}
        \end{tikzpicture}&
        \begin{tikzpicture}
            \begin{axis}[
                title = Testing Time,
                ybar,
                xtick=data,
                xticklabels from table = {\ourslargesummary}{Algorithm},
                x tick label style = {rotate=45, anchor=east},
                error bars/y dir=both,
                error bars/y explicit,
                error bars/error bar style={black},
                ylabel=Time/s,
                no markers,
                ymin=0,
                width=0.49\textwidth,
                bar width=6.pt,
            ]
                \addplot+[
                    draw=black,
                    fill=ourblue,
                ] table[
                    col sep=comma,
                    x expr=\coordindex,
                    y=test_time_avg,
                    y error=test_time_std,
                ] {\ourslargesummary};
                \addplot+[
                    draw=black,
                    fill=ourorange,
                ] table[
                    col sep=comma,
                    x expr=\coordindex,
                    y=test_time,
                ] {data/vgg16_17_flowers_summary.csv};
            \end{axis}
        \end{tikzpicture}\\
        \begin{tikzpicture}
            \begin{axis}[
                title = Training Loss,
                ylabel = Loss,
                ybar,
                xtick=data,
                xticklabels from table = {\ourslargesummary}{Algorithm},
                x tick label style = {rotate=45, anchor=east},
                error bars/y dir=both,
                error bars/y explicit,
                error bars/error bar style={black},
                ymin=0,
                width=0.49\textwidth,
                bar width=6pt,
            ]
                \addplot+[
                    draw=black,
                    fill=ourblue,
                ] table[
                    col sep=comma,
                    x expr=\coordindex,
                    y=train_loss_avg,
                    y error=train_loss_std,
                ] {\ourslargesummary};
                \addplot+[
                    draw=black,
                    fill=ourorange,
                ] table[
                    col sep=comma,
                    x expr=\coordindex,
                    y=train_loss,
                ] {data/vgg16_17_flowers_summary.csv};
            \end{axis}
        \end{tikzpicture}&
        \begin{tikzpicture}
            \begin{axis}[
                title = Validation Loss,
                ylabel = Loss,
                ybar,
                xtick=data,
                xticklabels from table = {\ourslargesummary}{Algorithm},
                x tick label style = {rotate=45, anchor=east},
                error bars/y dir=both,
                error bars/y explicit,
                error bars/error bar style={black},
                ymin=0,
                width=0.49\textwidth,
                bar width=6pt,
            ]
                \addplot+[
                    draw=black,
                    fill=ourblue,
                ] table[
                    col sep=comma,
                    x expr=\coordindex,
                    y=val_loss_avg,
                    y error=val_loss_std,
                ] {\ourslargesummary};
                \addplot+[
                    draw=black,
                    fill=ourorange,
                ] table[
                    col sep=comma,
                    x expr=\coordindex,
                    y=val_loss,
                ] {data/vgg16_17_flowers_summary.csv};
            \end{axis}
        \end{tikzpicture}\\
        \begin{tikzpicture}
            \begin{axis}[
                title = Test Accuracy,
                ylabel = Loss,
                ybar,
                xtick=data,
                xticklabels from table = {\ourslargesummary}{Algorithm},
                x tick label style = {rotate=45, anchor=east},
                error bars/y dir=both,
                error bars/y explicit,
                error bars/error bar style={black},
                ymin=0,
                width=0.49\textwidth,
                bar width=6pt,
            ]
                \addplot+[
                    draw=black,
                    fill=ourblue,
                ] table[
                    col sep=comma,
                    x expr=\coordindex,
                    y=test_acc_avg,
                    y error=test_acc_std,
                ] {\ourslargesummary};
                \addplot+[
                    draw=black,
                    fill=ourorange,
                ] table[
                    col sep=comma,
                    x expr=\coordindex,
                    y=test_acc,
                ] {data/vgg16_17_flowers_summary.csv};
            \end{axis}
        \end{tikzpicture}
    \end{tabular}
    \caption{Comparison a comparison of the different color constancy methods on the 102 flower dataset using our model (blue) and VGG16 (orange).}
    \label{fig:comparison_102_flowers}
\end{figure}  % Results of the experiment
\section{Conclusion}

We found no substantial gain in categorization accuracy in our experiments.
However, this does not necessarily imply that this strategy is pointless.

Our datasets were mostly composed of images acquired
outdoors in natural lighting settings. It is possible that datasets covering a
broader variety of illumination conditions will produce potential improvements.
As such, further research is needed to determine the usefulness of this strategy in different
lighting circumstances.


\section{Future Work}

In this work, we only investigated simple statistical methods as well as a complex model to serve as a representative
for deep learning based approaches (FC\textsuperscript{4}). While this can serve to deliver baseline results, it is worth investigating
whether other, more sophisticated methods of applying \gls{CC} will deliver better results.

Similar to the above concerns, in this paper we limited ourselves to only working with images of flowers. It is possible that images from
other domains will experience significantly different results. Most images of flowers are captured outside, meaning there is already less variation in
light color than pictures taken in other circumstances. Therefore, we believe that a followup study with a wider range of categories could be useful.
  % Conclusion and outlook

\bibliography{references}
\end{document}