\documentclass{bmvc2k}

\bmvcreviewcopy{??}

\title{An investigation of Color Constancy for Image Classification}

\addauthor{Uzairu Abubakar}{uzairu.abubakar@etu.univ-st-etienne.fr}{1}
\addauthor{Niklas Bieck}{niklas.bieck@etu.univ-st-etienne.fr}{1}
\addauthor{Fabiano Maia Junior Manschein}{fabiano.junior.maia.manschein@etu.univ-st-etienne.fr}{1}
\addauthor{Yuya Takagi}{yuya.takagi@etu.univ-st-etienne.fr}{1}

\addinstitution{
    Université Jean Monnet Saint-Étienne \\
    10, rue Tréfilerie – CS 82301 \\
    42023 Saint-Etienne Cedex 2 \\
    France
}

\runninghead{Abubakar, Bieck, Maia Manschein, Takagi}{Color Constancy Classification}

\begin{document}

\maketitle

\begin{abstract}
Color Constancy, attempting to determine the color of the light in a particular image, and then 
modifying the color to match what it would have been under some reference illuminant, is a widely studied topic.
One of the main claims that is made as to whether this is a useful endeavor is that it could aid in the training
of other image processing, as the light color is removed as a point of variation that needs to be learned.

This seems to just be taken as accepted fact and has never actually been empirically verified. In this paper, we
seek to rectify this situation by comparing the performance and training times of image classification models on 
a corpus of images of flowers (chosen because of the importance of color in their identification), with or without
having applied Color Constancy methods as preprocessing.
\end{abstract}

\bibliography{references}
\end{document}