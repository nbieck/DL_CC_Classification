\documentclass{bmvc2k}

%\bmvcreviewcopy{??}

\title{An Investigation of Colour Constancy for Image Classification}

\addauthor{Uzairu Abubakar}{uzairu.abubakar@etu.univ-st-etienne.fr}{1}
\addauthor{Niklas Bieck}{niklas.bieck@etu.univ-st-etienne.fr}{1}
\addauthor{Fabiano Junior Maia Manschein}{fabiano.junior.maia.manschein@etu.univ-st-etienne.fr}{1}
\addauthor{Yuya Takagi}{yuya.takagi@etu.univ-st-etienne.fr}{1}
\addauthor{Damien Muselet}{damien.muselet@univ-st-etienne.fr}{1}

\addinstitution{
    Université Jean Monnet Saint-Étienne \\
    10, rue Tréfilerie – CS 82301 \\
    42023 Saint-Etienne Cedex 2 \\
    France
}

\runninghead{Abubakar, et al.}{Colour Constancy Classification}

\begin{document}

\maketitle

\begin{abstract}
Colour Constancy, attempting to determine the colour of the light in a particular image, and then 
modifying the colour to match what it would have been under some reference illuminant, is a widely studied topic.
One of the main claims that is made as to whether this is a useful endeavor is that it could aid in the training
of other image processing, as the light colour is removed as a point of variation that needs to be learned.

We seek to validate this point by comparing the performance and training times of image classification models on 
a corpus of images of flowers (chosen because of the importance of colour in their identification), with or without
having applied Colour Constancy methods as preprocessing.
\end{abstract}

\section{Introduction}

\gls{CC} is a natural part of the human visual system that lets us recognize that a color
stays the same even under changing illumination. Letting computers do this same thing and then
modify the image color to represent their appearance under a neutral light source is a well-established
and on-going area of research. One of the presumed benefits of doing this is improved performance
of recognition tasks that rely to a large part on color, as it removes the need for the classification
model to learn to recognize objects under changing lighting conditions.

While some effort has been made to investigate how \gls{CC} correction can aid in video tracking
\cite{Agarwal2006}, as far as we can tell, no investigation specific to the field of object
classification has, as of yet, been performed. In this paper, we will do just that, using the
classification of flowers as a sample task, since this is clearly an area where color is of large
importance to a successful classification.

% Todo:
%! 1- lots of use of 'this'. Can the sentence be rewritten to be clearer?
%! 2- the use of terms such as 'as far as we know' and 'clearly' is dangerous. Alternatives?
%! 3- 'While some effort has been made...' -> 'While efforts to investigate... have been made, as far as...'

\section{Related Work}

As we do not propose any kind of new algorithm, we will make use of existing solutions for both
the computation of the colour corrected image, and the actual classification of the images.
On the side of colour constancy, we have found the follwing approaches, which include code/pretrained models:
\cite{hu2017fc} {\it find more here}.

There are two datasets of labeled images of flowers we can use to test: \cite{Nilsback06} and \cite{Nilsback08}


\bibliography{references}
\end{document}