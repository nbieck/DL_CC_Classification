\section{Future Work}

\subsubsection{Color Constancy Algorithms}

In this work, we only investigated very simple statistical methods as well as one more complex model to serve as a representative
for deep learning based approaches (FC\textsuperscript{4}). While this can serve to deliver baseline results, it might be worth investigating
whether other, more sophisticated method of applying color constancy will deliver better results.

\subsubsection{Datasets}

Similar to the above concerns, in this paper we limited ourselves to only working with images of flowers. It is possible that images from 
other domains will experience wildly different results. Most images of flowers are captured outside, meaning there is already less variation in
light color than pictures taken in other circumstances. We thus believe that a followup with a wider range of categories would be useful.

\subsubsection{Model Architecture}

Finally, in the interest of expediency we limited ourselves to only the use of two models, both of which are rather simple in architecture.
Since we did already observe a significant improvement when using the more complex model of the two (VGG16), this begs the question whether
more sophisticated models might see even more improvements, or whether the increased complexity of the classifier will negate some of the benefits
of this approach.

\section{Conclusion}

{\color{red}High-level summary here}