\section{Future Work}

\subsubsection{Color Constancy Algorithms}

In this work, we only investigated very simple statistical methods as well as one more complex model to serve as a representative
for deep learning based approaches (FC\textsuperscript{4}). While this can serve to deliver baseline results, it might be worth investigating
whether other, more sophisticated method of applying \gls{CC} will deliver better results.

\subsubsection{Datasets}

Similar to the above concerns, in this paper we limited ourselves to only working with images of flowers. It is possible that images from
other domains will experience wildly different results. Most images of flowers are captured outside, meaning there is already less variation in
light color than pictures taken in other circumstances. We thus believe that a followup with a wider range of categories would be useful.


\section{Conclusion}

In our experiments we were unable to find any partiular benefit to classification accuracy. That does not necessarily mean that this approach is a dead
end. As mentioned above, our datasets both consisted of images generally taken outside in natural light. It is possible that datasets captured under
more varied lighting conditions could actually see some improvements made.