\section{Conclusion}

We found no substantial gain in categorization accuracy in our experiments.
However, this does not necessarily imply that this strategy is pointless.

Our datasets were mostly composed of images acquired
outdoors in natural lighting settings. It is possible that datasets covering a
broader variety of illumination conditions will produce potential improvements.
As such, further research is needed to determine the usefulness of this strategy in different
lighting circumstances.


\section{Future Work}

In this work, we only investigated simple statistical methods as well as a complex model to serve as a representative
for deep learning based approaches (FC\textsuperscript{4}). While this can serve to deliver baseline results, it is worth investigating
whether other, more sophisticated methods of applying \gls{CC} will deliver better results.

Similar to the above concerns, in this paper we limited ourselves to only working with images of flowers. It is possible that images from
other domains will experience significantly different results. Most images of flowers are captured outside, meaning there is already less variation in
light color than pictures taken in other circumstances. Therefore, we believe that a followup study with a wider range of categories could be useful.
