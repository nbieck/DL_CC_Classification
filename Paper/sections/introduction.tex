\section{Introduction}

\gls{CC} is a natural characteristic of the human visual system that allows us to sense
that a color remains consistent despite changes in lighting. Extending this capability to computers
and then modifying image color to represent appearance under a neutral light source is a well-established
and on-going area of research \cite{FOSTER2011674}. One of the anticipated benefits of this technique is improved performance
in color-based recognition tasks, as it reduces the need for the classification model to learn object
recognition under different lighting situations \cite{Kasaei_Ghorbani_Schilperoort_vanderRest_2021}.

While some effort has been made to investigate how \gls{CC} correction can aid in video tracking
\cite{Agarwal2006}, as far as we can tell, no investigation specific to the field of object
classification has, as of yet, been performed. In this paper, we will do just that, using the
classification of flowers as a sample task, since this is clearly an area where color is of large
importance to a successful classification \cite{LU2021110082}.
