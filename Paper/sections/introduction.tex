\section{Introduction}

Colour constancy is a natural part of the human visual system that lets us recognize that a colour
stays the same even under changing illumination. Letting computers do this same thing and then 
modify the image color to represent their appearance under a neutral light source is a well-established
and on-going area of research. One of the presumed benefits of doing this is improved performance 
of recognition tasks that rely to a large part on colour, as it removes the need for the classification
model to learn to recognize objects under changing lighting conditions.

While some effort has been made to investigate how colour constancy correction can aid in video tracking
\cite{Agarwal2006}, as far as we can tell, no investigation specific to the field of object
classification has, as of yet, been performed. In this paper, we will do just that, using the
classification of flowers as a sample task, since this is clearly an area where color is of large
importance to a succesfull classification.